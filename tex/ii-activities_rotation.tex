\section{Rotation and Equilibrium}

\subsection{Centre of Gravity}

\subsubsection*{Learning Objectives}
\begin{itemize}
\item{To determine the centre of gravity of uniform wooden rod} 
\end{itemize}

\subsubsection*{Background Information}
The weight of object is concentrate at single point. This point is called the centre of gravity. For a uniform rod, the centre of gravity is normally at the centre of the rod.  Finding the center of gravity is something used often in daily life -- especially when balancing a bucket or large stick on the head.

\subsubsection*{Materials}
Uniform wooden rod about 1 m long, triangular blocks, ruler, wooden block

\subsubsection*{Activity Procedure}
\begin{enumerate}
\item{Place the wooden block on the table.} 
\item{Place the triangular block on the table so that the sharp edge is pointing upward.} 
\item{Slowly place the wooden rod on top of sharp edge of the triangle block and move it until it balances horizontally.} 
\item{Mark this balancing point on the wooden rod and measure its distance from both ends.} 
\end{enumerate}

\begin{figure}
\begin{center}
\def\svgwidth{175pt}
\input{./img/centre-of-gravity.pdf_tex}
\caption{Finding the Centre of Gravity}
\label{fig:center-of-gravity}
\end{center}
\end{figure}

\subsubsection*{Results and Conclusion}
The wooden bar balances near the middle. The balance point is known as the centre of gravity.  

\subsubsection*{Clean Up Procedure}
\begin{enumerate}
\item{Collect all materials and return them to their proper places.} 
\end{enumerate}

\subsubsection*{Discussion Questions}
\begin{enumerate}
\item{Was the centre of gravity at the centre of the wooden rod? How do you know?}
\item{Discuss another method to locate the centre of gravity of a uniform rod.} 
\end{enumerate}

\subsection{Verification of the Principle of Moments}

\subsubsection*{Learning Objectives}
\begin{itemize}
\item{To determine the moment of forces of a uniform wooden bar}
\end{itemize}

\subsubsection*{Background Information}
The body will be in equilibrium when the clockwise moments equal the anticlockwise moments. The turning forces depend on the product of the distance from the pivot to the point of application of the force and the magnitude of the perpendicular forces acting on the mass.  

\subsubsection*{Materials}
metre rule, three dry cells or any two objects of equal weight, thread, triangular wooden block or any object that will create a pivot point.  

\begin{figure}[h]
\begin{center}
\def\svgwidth{200pt}
\input{./img/equilibrium-moment.pdf_tex}
\caption{Equilibrium of Moments}
\label{fig:equilibrium-moment}
\end{center}
\end{figure}

\subsubsection*{Activity Procedure}
\begin{enumerate}
\item{Balance the metre rule on the pivot point so that it remains horizontal.} 
\item{Place two equal weights 20 cm from the pivot on the right and left sides of the pivot so that the ruler remains balanced.} 
\item{Now move the right weight 5 cm further away from the pivot point and observe what happens.} 
\item{Return that weight to the 20 cm mark and the system back to equilibrium. Now move the same weight 5 cm closer to the pivot point and observe what happens.} 
\item{Now balance the metre rule itself on the pivot and place one weight at the 20~cm on the left side of the pivot and the two other weights at the 10~cm mark on the opposite side of the pivot. The metre rule should remain balanced.} 
\end{enumerate}

\subsubsection*{Results and Conclusion}
When the system is in equilibrium the product of the distance to the pivot point and the weight on opposite sides of the pivot are equal.  

\subsubsection*{Clean Up Procedure}
\begin{enumerate}
\item{Put the instruments back in their respective places.} 
\end{enumerate}

\subsubsection*{Discussion Questions}
\begin{enumerate}
\item{Explain the relation between the masses and distance from the pivot when the metre rule is at equilibrium}
\end{enumerate}

\subsubsection*{Notes}
The wooden bar or metre rule should be uniform in order to simplify this experiment.
Many objects and tools operate by the principle of moments.  For a turning force to be more effective, the distance of the force from the pivot should be large.  A force close to the pivot will have a smaller effect.  This explains why the handle of a door is far from the hinge, or why a bottle opener has a long handle.
