\section{Conservation of Energy}

\subsection{Potential Energy of a Spring}

\subsubsection*{Learning Objectives}
\begin{itemize}
\item{To observe the change in energy from potential to kinetic}
\end{itemize}

\subsubsection*{Background Information}
In a closed system, where no force acts on the objects, the total energy remains constant.  In the case of mechanical energy, this means that potential energy and kinetic energy can change, but their total remains the same.  Springs and other elastic materials also have potential energy in the form of elastic potential.

\subsubsection{Materials} 
Clothes pin, thread, two pencils\\

\subsubsection{Activity Procedure}
\begin{enumerate}
\item{A clothes pin has two end: one for gripping clothes and one for pushing with fingers. Press the finger ends together so the gripping end is open as far as possible. Use a thread to tie the finger ends together so that the gripping ends stay wide open.}
\item{Place the clothes pin on a table and place two pencils parallel to the clothes pin, one on either side, with the tips on the gripping end and erasers near the thread. Adjust the pencils so that the sides are touching the sides of the clothes pin. Adjust the clothes pins so that the erasers just meet behind the cloths pin. The pointed ends should extend beyond the front of the gripping end of the clothes pin.}
\item{Cut the thread holding the clothes pin open.}
\end{enumerate}

\subsubsection*{Cleanup Procedure}
\begin{enumerate}
\item{Return all materials to their proper places.}
\end{enumerate}

\subsubsection*{Discussion Questions}
\begin{enumerate}
\item{What two types of energy are being used here?}
\item{Describe the change in energy occurring here.}
\end{enumerate}
 
\subsubsection{Notes}
The spring inside the clip holds energy when it is forced to contract.  
When the clip is allowed to close, the potential energy of the spring
is converted into mechanical energy as the clip moves, forcing the pencils
away quickly.

The interconversion of potential and kinetic energy may also be shown with a simple pendulum. Another option is to hold a heavy book above the table and then to drop it on the table.
