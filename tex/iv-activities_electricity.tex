\section{Electricity}

\subsection{Creating a Leclanche Cell.}

\subsubsection*{Learning Objectives}
\begin{itemize}
\item{To describe the production of electric current by an electrochemical cell}
\item{To construct a simple electrochemical cell}
\end{itemize}

\subsubsection*{Background Information}
A cell converts chemical energy to electrical energy.  We know cells as batteries or accumulators.

\subsubsection*{Materials}
Many lemons, zinc plate and carbon rod from a dead dry cell battery, connecting wires, galvanometer, bulb

\subsubsection*{Hazards and Safety}
\begin{itemize}
\item{The black powder found in the dry cell is poisonous. The black powder will also corrode metal -- wash all tools well that touch the powder.}
\item{The outer case of the dry cell battery can be quite sharp -- take care when opening.}
\end{itemize}

\subsubsection*{Preparation Procedure}
\begin{enumerate}
\item{Open a dry cell battery and extract the inner shell of zinc metal and the central carbon rod.}
\end{enumerate}

\subsubsection*{Activity Procedure}
\begin{enumerate}
\item{Make two holes in each lemon then insert the carbon rod and zinc plate in the holes.} 
\item{Connect one lemon with the galvanometer and then add more in series. Note changes in the galvanometer.}
\item{Arrange the lemons in series with the switch and bulb using connecting wire.} 
\end{enumerate}

\subsubsection*{Results and Conclusion}
With one lemon connected to the galvanometer a deflection will occur. The deflection will increase with the number of lemons arranged in series. After completing the circuit with enough lemons, the bulb will light up. This shows that a lemon can be used to produce a wet cell known as Leclanche cell.  

\subsubsection*{Clean Up Procedure}
\begin{enumerate}
\item{Collect lemons and throw them in the dust bin. Make sure that people and animals do not eat the used lemons -- they will contain poisonous residual manganese (IV) oxide from the battery and an unhealthy amount of zinc.}
\item{Put other materials in their respective places}
\end{enumerate}

\subsubsection*{Discussion Questions}
\begin{enumerate}
\item{Explain you observations of the completed circuit.}
\item{In the absence of lemons what other materials can be used to create the Leclanche cells?} 
\end{enumerate}

\subsubsection*{Notes}
Electric current can be produced from different sources(cells). There are dry and wet sources of electric cells. Wet cells can be made from natural fruits and foods such as lemon, Irish potatoes and salts which produce electric current based on the principle of Leclanche cells.


\subsection{Construction of a Metre bridge and Potentiometer}

\subsubsection*{Learning Objectives}
\begin{itemize}
\item{To construct and use a metre bridge and potentiometer} 
\end{itemize}

\subsubsection*{Background Information}
A metre bridge can be used to determine the resistance of various conductors and also to compare the Emf of batteries. It consists of a piece of resistance wire 1 m long attached in series to either a rheostat or other conductors and resistors. By using a galvanometer we can find the length of resistance wire needed to equal the resistance of the other conductors. Or we can use a voltmeter to find the potential difference along a length of the resistance wire. There are several activities that can be done with a metre bridge and each requires different materials. The construction of the metre bridge itself, however, is the same, and follows below.

\subsubsection*{Materials}
Piece of wood 110 cm by 4 cm, shoe tacks or small nails or screws, connecting wire, Constantine or nichrome wire, dry cells, resistors of known and unknown resistance, galvanometer. Alternative: Voltmeter, rheostat, micrometer screw gauge

\subsubsection*{Hazards and Safety}
\begin{itemize}
\item{Make sure all nails are removed.} 
\end{itemize}

\subsubsection*{Preparation Procedure}
\begin{enumerate}
\item{From one side of the piece of wood, measure 5~cm and fix 2 shoe tacks or nails side by side about 2.5~cm apart, as shown in \ref{fig:metre-bridge}. From one of the nails measure 10~cm and fix another nail. Leave a space of about 6-8~cm and fix another nail. From the first nail again measure 50~cm and fix another nail.} 
\item{Again from the first nail measure 100~cm and fix two nails apart as with the first end. From this end on the side of one of the nails measure 10~cm and fix another nail, leave a gap and fix another nail.} 
\item{On the other long side place a meter ruler from first nail to the last nail, mark each cm and write the scale interval of 10~cm.} 
\item{Fix Constantine wire from the first nail to last nail along the side with the cm markings and make sure the wire is tight.} 
\end{enumerate}

\begin{figure}
\begin{center}
\def\svgwidth{350pt}
\input{./img/metre-bridge.pdf_tex}
\caption{A Metre Bridge}
\label{fig:metre-bridge}
\end{center}
\end{figure}

\subsubsection*{Activity Procedure}
\begin{enumerate}
\item{Use connecting wire to join one of the nails holding the resistance wire to the second nail along the side without the cm markings. Then leave a gap and connect another wire to the central nail and then to next nail. Leave another gap and connect a wire from following nail to nail holding the other end of the resistance wire.} 
\item{Connect a metre bridge circuit by connecting one known and one unknown resistor into each of the gaps.} 
\item{Connect one terminal of a galvanometer to the center nail of the metre bridge. Connect the other terminal to a connecting wire which can be moved easily along the resistance wire.} 
\item{Move the galvanometer wire along the resistance wire until the galvanometer reads zero; that is no current is flowing.} 
\item{Measure the length of resistance wire on both sides of the galvanometer wire. Call the left side of the metre bridge ``one'' and the right side ``two'' so these lengths are $L_1$ and $L_2$.} 
\item{Write these two lengths as a ratio, for example the length on the left side divided by the length on the right side, $\frac{L_1}{L_2}$.} 
\item{Set this ratio equal to the ratio of resistances of the resistors, in this case the resistor on the left divided by the resistor on the right (one of these resistances is know, one is not).
$$ \frac{L_1}{L_2} = \frac{R_1}{R_2} $$ 
}
\item{Solve this equation to find the resistance of the unknown resistor, that is $$ R_1 = R_2 \frac{L_1}{L_2} $$ or $$ R_2 = R_1 \frac{L_2}{L_1} $$
} 
\end{enumerate}

\begin{figure}
\begin{center}
\def\svgwidth{300pt}
\input{./img/ohm's-law.pdf_tex}
\caption{Metre bridge circuit}
\label{fig:ohm's-law}
\end{center}
\end{figure}

\subsubsection{Alternative Activity: Potentiometer}
\begin{enumerate}
\item{In the gaps of the metre bridge, place a rheostat, dry cells and a switch.} 
\item{Connect one terminal of a voltmeter to the nail holding the resistance wire at the 0 cm mark.} 
\item{Connect the other terminal of the voltmeter to a connecting wire which can easily move along the resistance wire.} 
\item{Close the switch so that current is moving through the circuit.} 
\item{Move the voltmeter wire to the 10 cm mark and read the voltage. This is the potential difference across 10 cm of the resistance wire.} 
\item{Move the voltmeter wire to the 20 cm mark and read the voltage. Repeat this process for 30, 40, 50, etc. cm and record all of the values.} 
\item{Use the callipers to measure the diameter of the wire.} 
\item{Make a graph of resistance and length. Use this graph to find the resistivity of the resistance wire.} 
\end{enumerate}

\begin{figure}
\begin{center}
\def\svgwidth{300pt}
\input{./img/potential-metre.pdf_tex}
\caption{Layout of a Potentiometer}
\label{fig:potential-metre}
\end{center}
\end{figure}

\subsubsection*{Results and Conclusion}
A constructed metre bridge functions using the relationship between a wire's length and its resistance. This principle can be used to find the resistivity of the wire or to find the resistance of any other conductors.  

\subsubsection*{Clean Up Procedure}
\begin{enumerate}
\item{Remove any remaining nails, pieces of wood and other equipment.} 
\end{enumerate}

\subsubsection*{Discussion Questions}
\begin{enumerate}
\item{How can you manufacture a potentiometer?}
\end{enumerate}

\subsubsection*{Notes}
Metre bridges are used to determine unknown resistance and to compare electromotive force.  When you are using a metre bridge to find the resistance of an unknown resistor, be sure to use two resistors that are similar in resistance. If the resistances are too different, it will be difficult to find with precision the point on the wire where the galvanometer reads zero.

This setup can also be used as a potential meter (potentiometer) in order to find the resistivity of the resistance wire.  

\subsection{Creating a Light Bulb}

\subsubsection*{Learning Objectives}
\begin{itemize}
\item{To observe the luminosity effect of current in a thin wire}
\item{To understand the mode of action of a light bulb}
\end{itemize}

\subsubsection*{Background Information}
When electricity pass through a wire, resistance converts electrical energy into heat energy. This heat energy causes an increase in the temperature of the wire. If the wire gets hot enough, it will release energy as radiation visible light, red at lower temperatures to yellow to white at very hot temperatures. The effect, however, causes the wire to degrade and eventually the wire is not strong enough to pass current.  When a lot of electrical energy is converted into light energy, the effect is that of a light bulb, which consists of a single, thin wire in a glass bulb.

\subsubsection*{Materials}
Glass jar with lid, glue, wires, power source, small length of thin iron wire, nail

\subsubsection*{Preparation Procedure}
\begin{enumerate}
\item{Use the nail to poke two holes in the jar lid.}
\item{Pass a wire through each hole about half way into the jar.}
\item{Connect the wires inside the jar with the length of iron wire.}
\item{Seal the wires into the lid with glue.}
\item{Close the lid on the jar.}
\end{enumerate}

\subsubsection*{Activity Procedure}
\begin{enumerate}
\item{Connect the wires outside the jar to the power source.}
\item{Observe the effect inside the jar.}
\end{enumerate}

\subsubsection*{Results and Conclusion}
If enough current is passing, the iron wire will light up, creating a light bulb for a short time until the
wire burns out.  

\subsubsection*{Cleanup Procedure}
\begin{enumerate}
\item{Disconnect all wires and return all materials to their proper places.}
\end{enumerate}

\subsubsection*{Discussion Questions}
\begin{enumerate}
\item{Why does the wire produce light?}
\item{Why do bulbs eventually stop working?}
\item{Mention some other materials which produce light when heated.}
\end{enumerate}

\subsubsection*{Notes}
You may need to try different types of wire for the bulb.  If iron wire is not available or does not work, try other types.  It should be very thin and have high resistance in order to work.

\subsection{Fuse}

\subsubsection*{Learning Objectives}
\begin{itemize}
\item{To understand the effect of a strong electric current on a wire of high resistance}
\item{To understand the use of a fuse in electrical devices and installations}
\end{itemize}

\subsubsection*{Background Information}
When a strong current is passed through a wire of high resistance (depending on the material, area and length), some of the electrical energy is converted into heat energy.  If enough heat is produced, the wire burns and stops conducting electricity, breaking the circuit.  This device can be used to protect electrical devices like radios or computers from electrical surges.  If placed before all the components in a circuit, this thin wire, called a fuse, will break if too much current is passed through the circuit.

\subsubsection*{Materials}
Power source, wires, two small nails, small piece of wood, metal foil (from Blueband container, wrapper, etc.)

\subsubsection*{Preparation Procedure}
\begin{enumerate}
\item{Hammer the nails into the wood about 5 cm apart.}
\item{Connect wires to each of the nails.}
\item{Place a thin strip of foil between the nails, bending it around the nails to secure it.}
\end{enumerate}

\subsubsection*{Activity Procedure}
\begin{enumerate}
\item{Connect the wires to a power source to complete the circuit.}
\item{Observe the effect of the current on the foil when a large current is passing.}
\end{enumerate}

\subsubsection*{Results and Conclusion}
If the source is powerful enough, it will cause the foil to heat and eventually burn, breaking the circuit.  It will be seen that the circuit will work at first, but then the foil will burn.  Foil has a very small cross-sectional area compared to that of a wire, so it has a low tolerance for current. If too much current passes through the foil, it will burn away. This is essentially how a fuse works in a radio or other electrical device.

\subsubsection*{Cleanup Procedure}
\begin{enumerate}
\item{Disconnect the wires from the battery.}
\item{Dispose of the burnt fuse and return all materials to their proper places.}
\end{enumerate}

\subsubsection*{Discussion Questions}
\begin{enumerate}
\item{What causes the fuse to burn?}
\item{Why would a fuse be useful in an electrical device?}
\end{enumerate}

\subsubsection*{Notes}
It is best to use a rheostat in this circuit.  Begin with a high resistance on the rheostat; slowly decrease it so that the foil is taking more of the voltage.  As the voltage is increased, the foil will eventually burn as its resistance reaches a maximum.
You can ind fuses in most electrical devices; show these in class after performing this activity.


\subsection{Making an Electric Heater}

\subsubsection*{Learning Objectives}
\begin{itemize}
\item{To prepare and demonstrate an electric heater} 
\end{itemize}

\subsubsection*{Background Information}
Electricity and heat are both forms of energy, so it is possible to change energy from one form to another.  

\subsubsection*{Materials}
1 m of resistance wire (nichrome) of swg 30 or 32, small piece of wood or paper, connecting wires, two to four dry cells (batteries), container of water, optional thermometer

\subsubsection*{Hazards and Safety}
\begin{itemize}
\item{If the heater is connected to the cells while not in the water, the wire can melt or burn other objects.} 
\end{itemize}

\subsubsection*{Preparation Procedure}
\begin{enumerate}
\item{Collect all materials.} 
\item{Cut a small piece of wood about 4 to 6 cm long, or roll a piece of paper.} 
\end{enumerate}

\subsubsection*{Activity Procedure}
\begin{enumerate}
\item{Coil the resistance wire onto the piece of wood or rolled paper so that the coils are close but not touching. Use the entire wire.} 
\item{Use connecting wires to connect the ends of the resistance wire to the terminals of the batteries.} 
\item{Place the coil of resistance wire into the container of water.} 
\item{Observe any effects on the water when current is flowing through the coil. Use your hand on the outside of the container to test the temperature.} 
\end{enumerate}

\subsubsection*{Results and Conclusion}
When the heater (coil) is left in the container of water for a few minutes, the temperature of the water rises enough that it can be felt by your hand on the outside of the container. If the heater is left for a long time, the water may boil.  

\subsubsection*{Clean Up Procedure}
\begin{enumerate}
\item{Remove the waste products and dry the water on the table.} 
\item{Return all materials to their proper places.} 
\end{enumerate}

\subsubsection*{Discussion Questions}
\begin{enumerate}
\item{What kind of energy is being produced?}
\item{What kind of energy is being converted?}
\end{enumerate}

\subsubsection*{Notes}
The electric heater converts electrical energy into heat energy. It can be used to boil water or other liquids, or to heat houses. The larger the coils are, the more efficient the heater will be.  


\subsection{Inverter: Converting DC to AC}

\subsubsection*{Learning Objectives}
\begin{itemize}
\item{To explain the difference between Alternating Current and Direct Current} 
\item{To understand how to convert Direct Current to Alternating Current by changing the directing of current through a bulb or galvanometer} 
\item{To observe the effect of alternating current on a bulb or galvanometer} 
\end{itemize}

\subsubsection*{Background Information}
Direct Current is the current produced by any battery, cell or generator. It is current which moves in one direction only through a circuit. Alternating current is the current used in a house, school, etc. to power appliances, charge batteries, etc. It is current which changes direction many times per second; in Tanzania that current changes direction 80 times per second, so we say that the Tanzanian electrical system runs on 80 Hz, or 80 cycles per second.  
Rectifiers are used to convert AC to DC, but to convert the other way we need an inverter which can change the direction of electric current many times per second. The easiest way to do this is to switch the terminals of the battery repeatedly.  

\subsubsection*{Materials}
4 dry cells in the plastic wrapping, Aluminium outer coating of a dead dry cell, thin cardboard, super glue, soldering iron and flux, several small nails or screws, connecting wires, board at least 1 ft long, bulb or galvanometer, scissors, knife, retort stand with clip or alternative, small motor, horizontal pulley, rubber band, pliers, multimeter


\subsubsection*{Preparation Procedure}
\begin{enumerate}
\item{Collect all of the necessary supplies.} 
\item{Attach the horizontal pulley near one end of the board so that it is free to rotate horizontally.} 
\item{Attach the small motor to the board so that it is also free to rotate horizontally.} 
\item{Attach the rubber band to the motor and pulley so that when the motor turns, the pulley also turns.} 
\item{Connect two connecting wires to the terminals of the motor and make sure it works by connecting two dry cells.} 
\item{On one of the dry cell packs, cut away the plastic around each of the terminals on both batteries so that the batteries can be used without removing them from the plastic.} 
\item{Solder or glue a connecting wire from the positive terminal of one of the batteries in the plastic to the negative terminal of the other battery. If needed, glue the wire down to the plastic in the middle so that it does not stick up.} 
\item{Glue the center battery pack on its side to the center of the horizontal pulley so that when the pulley turns, the battery pack turns on its side also.} 
\item{Cut a small piece of thin cardboard to fit over the battery pack (about 5 cm square).} 
\item{Glue the thin cardboard to the top of the battery pack.} 
\item{Find the exact center of rotation of the battery pack and pulley by rotating the pulley and marking the center of rotation on the cardboard with a pen.} 
\item{Remove the aluminium from the outside of a dead dry cell.} 
\item{Break the aluminium into 2 equal pieces (about 5 cm x 3 cm) by bending it with straight pliers.} 
\item{Glue the aluminium pieces to the cardboard so that there is a small space between them (so electricity cannot pass) and the hole marking the center of rotation can be seen between them.} 
\item{Rotate the pulley again to make sure that, as it rotates, the metal plates are rotating about the axis.} 
\item{Solder or glue a connecting wire from the free end of one of the batteries in the battery pack to one of the aluminium plates. Be sure that the wire is short enough that it does not stick up or out too much.} 
\item{Repeat this step for the other free battery terminal and the other aluminium plate.} 
\item{Use a multimeter to make sure that all of the wire connections are complete and that the circuit can conduct from one plate through the battery pack to the other plate. Note that the circuit is not complete because of the thin gap between the metal plates (otherwise you will have a short).} 
\item{Cut a piece of thin cardboard about 10 cm long and 4 cm wide.} 
\item{Fold the cardboard in half the long way.} 
\item{Cut two very small holes in the center of the folded edge about 2 cm apart.} 
\item{Insert connecting wires into each of these holes so that each sticks out of the folded end by about 2 cm.} 
\item{Remove the insulating from these connecting wires so that the copper ends form brush shapes and are free to bend slightly.} 
\item{Extend the connecting wires back to a bulb or galvanometer and solder or glue the ends to the terminals.} 
\item{Check that this circuit works with a multimeter or by connecting 2 dry cells across the brush ends.} 
\item{If possible, connect a switch to the bulb or galvanometer.} 
\item{Suspend the bulb/brush circuit about the rotating metal plates so that the wire brushes just touch the metal plates. If each brush is touching a different plate, the bulb or galvanometer should indicate a current.} 
\end{enumerate}

\subsubsection*{Activity Procedure}
\begin{enumerate}
\item{Check the circuits in the inverter by turning on the motor or touching the wire brushes to the metal plates.} 
\item{Align the wire brushes so that they rest in the thin gap between the metal plates. This will allow you to continue working or discussing without using up the battery or bulb.} 
\item{Touch the wire brushes to opposite plates to show that a direct current is flowing and the bulb produces a steady light or the galvanometer shows a single direction of current.} 
\item{Switch the plates that the brushes are touching to show that, again, a direct current is flowing and the bulb produces a steady light or the galvanometer shows a single direction of current opposite to the one previously seen.} 
\item{Connect the motor to the batteries so that the system rotates under the brushes.} 
\end{enumerate}

\subsubsection*{Results and Conclusion}
Students will see that when the metal plates are rotating under the brushes, the bulb flickers on and off quickly or the galvanometer changes direction quickly.  
Students will see that the behavior of the bulb or galvanometer is different depending on if the plates are rotating.  
Students will understand that the bulb flickers or the galvanometer changes direction because the direction of current is changing every time the plates switch brushes.  
Students will understand that the system is converting the direct current (DC) of the battery pack to alternating current (AC) in the bulb or galvanometer.  

\subsubsection*{Clean Up Procedure}
\begin{enumerate}
\item{Disconnect the motor from the batteries and remove the bulb/brush circuit from the metal plates.} 
\item{Return all pieces to their proper places.} 
\end{enumerate}

\subsubsection*{Discussion Questions}
\begin{enumerate}
\item{What is the difference between direct and alternating current?}
\item{What is powering the bulb or galvanometer?}
\item{What is the purpose of rotating the metal plates under the wire brushes?}
\item{Why does the bulb flicker or the galvanometer repeatedly change direction when the metal plates are rotating under the wire brushes?}
\item{What type of current is powering the bulb or galvanometer when the metal plates are not rotating?}
\item{What type of current is powering the bulb or galvanometer when the plates are rotating under the brushes?}
\end{enumerate}

\subsubsection*{Notes}
Normally, alternating current changes direction 80 times per second, which we cannot see with our eyes. Therefore, the difference between AC and DC is not visible in a normal household or school electrical system. In order to see the effect of AC, we need to slow down the frequency to the point where we can see the direction changing in the galvanometer or bulb.
