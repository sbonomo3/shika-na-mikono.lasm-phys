\section{Forces}

\subsection{Presence of Gravity}

\subsubsection*{Learning Objectives}
\begin{itemize}
\item{To identify the force of gravity as it acts on falling bodies} 
\item{To identify the effect of air resistance on falling bodies} 
\end{itemize}

\subsubsection*{Background Information}
All objects on the earth experience a force of attraction exerted by the earth.  This is a natural force called Gravity and it acts on all bodies at all times.  The force of gravity varies from one point to another; some areas experience stronger gravity than others, but this effect is not noticeable.  All objects are pulled by gravity with equal force, regardless of their weights or masses.

\subsubsection*{Materials}
Various objects, a piece of paper, and a book (the book should be the same size or bigger than the paper)

\subsubsection*{Activity Procedure}
\begin{enumerate}
\item{Hold the various objects at shoulder height.} 
\item{Drop the objects to the ground one at a time. Repeat this step, but releasing the objects at the same time.} 
\item{Observe if there is any difference in speed as the objects fall to the ground.} 
\item{Hold a piece of paper at shoulder height and then release it.} 
\item{Place a piece of paper on top of a book and hold the book flat at shoulder height.} 
\item{Release the two items together and observe any differences between the motion of the paper by itself and of the paper and book together.} 
\item{Bunch the paper into a tight ball and drop it again.}
\end{enumerate}

\subsubsection*{Results and Conclusion}
It will be seen that all objects, with the exception of paper and other light, wide objects, fall at exactly the same rate. This is because the acceleration due to gravity for all objects on earth is the same.  
The paper, however, falls very slowly. This is not because gravity pulls less on paper; it is because the paper is more affected by air resistance. All objects are affected by air resistance, but it is most obvious with objects that have a small weight but a large surface area. The effects of air resistance can be greatly reduced by placing a book under the paper. The book moves easily through air and blocks all of the air which would normally push against the paper. This is why the paper and book fall at the same rate.  When the paper is bunched into a ball, the mass stays the same but the air resistance is greatly reduced so it should fall at the same rate as the book.

\subsubsection*{Clean Up Procedure}
\begin{enumerate}
\item{Collect all materials and return them to their proper place.} 
\end{enumerate}

\subsubsection*{Discussion Questions}
\begin{enumerate}
\item{Did the objects fall at the same rate or at different rates?}
\item{Why did the paper fall slowly the first time?}
\item{Why did the paper fall quickly when it was placed on top of the book?}
\item{Why did the paper fall quickly when it was bunched into a tight ball?}
\item{What force is pulling all objects down? Does this force ever change?}
\end{enumerate}

\subsubsection*{Notes}
When performing this experiment, it is important to remember the effect of air resistance.  Gravity pulls equally on all bodies, but air resistance opposes the motion of light-weight objects more effectively than heavy-weight objects.

\subsection{Making a Spring and a Spring Balance}

\subsubsection*{Learning Objectives}
\begin{itemize}
\item{To make springs for various uses.}
\item{To create and calibrate a spring balance.}
\end{itemize}

\subsubsection*{Background Information}
Springs typically have a constant value which determines how much they can stretch or compress.  Because the value is constant, we can use it to accurately measure mass or weight, as it will always extend to the same length with the same force.

\subsubsection*{Materials}
Strong metal wire of swg 24 or 26 of various types like copper, nichrome or constantine; a rod of diameter 14 to 18 mm of any material, piece of wood, ruler, plane paper, known masses, glue.

\subsubsection*{Preparation Procedure}
\begin{enumerate}
\item{Cut the metal wire into different lengths of at least 50 cm.}
\end{enumerate}

\subsubsection*{Activity Procedure}
\begin{enumerate}
\item{Take a piece of wire and hold one end to the rod.}
\item{Coil the wire tightly along the rod, keeping the coils close together.}
\item{When the entire wire is coiled along the rod, remove the rod.}
\item{Bend btoh ends of the wire into hooks.}
\item{Repeat these steps for all of the wires of different types and lengths.}
\item{Try stretching and compressing the springs to test the different strengths.}
\item{Attach the end of one spring to the top of a piece of wood with a nail or screw.}
\item{Glue or tape some white paper to the wood behind the spring.}
\item{Use a pen to make a mark at the bottom of the spring.}
\item{Hang a known mass on the spring so that it is stretched downward a short distance.}
\item{Make a mark at the bottom of the spring in its new position.  Label this with the mass of the object hung from the spring.}
\item{Repeat this process for other masses, marking each mass on the paper.}
\item{Use a ruler to finish the scale by filling in masses above, below and between the marks you have made.}
\end{enumerate}

\subsubsection*{Results and Conclusion}
When a load is placed on the spring, it increases the length of the spring.  When the load is removed, the spring returns to its original length.
Different materials of the same swg have different spring constants (strengths), or force per extension length.

\subsubsection*{Clean Up Procedure}
\begin{enumerate}
\item{Return all materials to their proper places.}
\end{enumerate}

\subsubsection*{Discussion Questions}
\begin{enumerate}
\item{What are some uses of springs?}
\item{What are the qualities of a good spring?}
\item{What is the relationship between the wire's diameter and the strength of the spring?}
\item{Which metal makes the strongest spring?  Which metal makes the weakest spring?}
\end{enumerate}

\subsubsection*{Notes}
A good spring must obey Hooke's Law.  Springs can be used to make a spring balance or can be used in physics practicals to find the relationship between force and the resulting extension.
