\section{Friction}

\subsection{Concept of Friction}

\subsubsection*{Learning Objectives}
\begin{itemize}
\item{To describe the methods of reducing friction} 
\end{itemize}

\subsubsection*{Background Information}
Friction is the force which opposes motion.  When two bodies in contact are moving against each other, there is opposition to the motion.  Friction between two surfaces depends on the nature of the surfaces in contact.

\subsubsection*{Materials}
lubricants*, rollers*, ball bearings, 50 cm wood board

\subsubsection*{Activity Procedure}
\begin{enumerate}
\item{Place the block of wood on the rollers.} 
\item{Slide the block and observe how easily it moves.} 
\item{Place the block of wood on the table without rollers and push it slightly.} 
\item{Smear a lubricant on the table top, then place the block of wood on the table and push it slightly.} 
\end{enumerate}

\subsubsection*{Results and Conclusion}
The block of wood will be easier to push on the rollers and lubricated table because these things reduce friction.  

\subsubsection*{Clean Up Procedure}
\begin{enumerate}
\item{Clean up the smeared surface with oil/grease.} 
\end{enumerate}

\subsubsection*{Discussion Questions}
\begin{enumerate}
\item{Why does a car slide on the road during a heavy rainstorm?}
\item{Which surface do you think has more friction, a tile floor or a dirt floor? Why?}
\end{enumerate}

\subsubsection*{Notes}
Friction causes noticeable effects; these effects can be reduced in class by using rollers, wheels and lubricants like oil and air.

\subsection{Limiting Friction and the Coefficient of Static Friction}

\subsubsection*{Learning Objectives}
\begin{itemize}
\item{To make use of an inclined plane to determine the limiting friction} 
\item{To observe the effect of static friction on a body} 
\item{To calculate the coefficient of static friction of a body on an inclined plane} 
\end{itemize}

\subsubsection*{Background Information}
Friction opposes the motion of all bodies. As a body slides along a surface, the surface itself opposes the motion of the body, causing a force in the opposite direction of the body's motion. This type of friction is called Dynamic Friction and depends on the nature of the surfaces in contact.  

Static friction, however, opposes the force on an object that is not moving. If you push an object gently, it will not move. In order to cause the object to move, you need to increase your force. The force which is opposing you is the static friction between the object and the surface it is resting on. Static friction is present until the object starts moving. At that point, the force of friction is called Limiting Friction: the minimum force needed to cause an object to move.  

\subsubsection*{Materials}
Two smooth wooden boards about 60~cm by 10~cm by 2~cm, wood block about 10~cm by 5~cm by 2~cm with one rough side and one smooth side, nails, hinge from a door or window, objects of different masses, thread, pulley, beam balance or spring balance (see the balance construction activities to make one yourself), protractor, plastic water bottle

\subsubsection*{Preparation Procedure}
\begin{enumerate}
\item{Connect the boards together at one end with a hinge and nails.} 
\item{Find a pulley from a broken tape deck or other source.} 
\item{Fix the pulley to the end of one of the boards (the end that is not attached to the hinge).} 
\item{Prop up the board with the pulley to create in inclined plane.} 
\item{Cut the bottom 5 cm off of the water bottle to create a scale pan.} 
\end{enumerate}

\end{enumerate}
\begin{figure}[h]
\begin{center}
\def\svgwidth{150pt}
\input{./img/friction-plank.pdf_tex}
\caption{Determining the Limiting Friction}
\label{fig:friction-plank}
\end{center}
\end{figure}

\subsubsection*{Activity Procedure}
\begin{enumerate}
\item{Use a balance to measure the mass of the wooden block. Record this value.} 
\item{Use a protractor to measure the inclination of the inclined plane. Do this by measuring the angle between the two boards at the hinge.} 
\item{Place the wood block on the inclined plane and attach one end of the thread to it.} 
\item{Pass the thread over the pulley fixed to the top of the inclined plane so that it hangs vertically.} 
\item{Attach the hanging thread to the plastic water bottle bottom (scale pan).} 
\item{Add objects to the scale pan until the block begins to slide up the inclined plane.} 
\item{Measure the mass in the scale pan and record this value.} 
\item{Increase the angle of the inclined plane and repeat these steps, measuring the mass needed to move the block in each case.} 
\item{Convert the masses from the scale pan to weight. These weights are equal to the limiting friction.} 
\item{Use these values and the weight of the block to calculate the coefficient of static friction between the block and the inclined plane.} 
\end{enumerate}

\subsubsection*{Results and Conclusion}
The mass, and therefore weight, needed to move the block increases when the angle increases. The weight is the force on the block, or the Limiting Friction. The limiting friction is the minimum force necessary to move the block; this is measured using the weights in the scale pan.  

The values for limiting friction can be used to calculate the coefficient of static friction. First we need the normal reaction on the block, which is the force perpendicular to the inclined plane. The normal reaction will be equal to the weight of the block times the cosine of the angle of the inclined plane.  

We find the coefficient of static friction by dividing the limiting friction by the normal reaction.

\begin{center}
$$ \mathrm{coefficient\;of\;static\;friction} = \frac{\mathrm{limiting\;friction}}{\mathrm{normal \;friction}} $$
\end{center}

\subsubsection*{Clean Up Procedure}
\begin{enumerate}
\item{Remove the masses and return all materials to their proper places.} 
\end{enumerate}

\subsubsection*{Discussion Questions}
\begin{enumerate}
\item{What is the limiting friction in each case?}
\item{Does the limiting friction increase or decrease as the angle increases?}
\end{enumerate}

\subsubsection*{Notes}
Limiting friction is used to find the coefficient of static friction. It is possible in this activity to calculate the coefficient of static friction is you first calculate the normal reaction, which will be equal to the weight of the wood block times the cosine of the angle of inclination of the inclined plane.
