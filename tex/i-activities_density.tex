\section{Density and Relative Density}

\subsection{Relative Density of a Liquid}

\subsubsection*{Learning Objectives}
\begin{itemize}
\item{To determine the density and relative density of a liquid.} 
\end{itemize}

\subsubsection*{Background Information}
Bodies of different materials have different densities.  Density can be found by taking the ratio of a body's mass to its volume.  EQUATION: Density = mass / volume
Relative density (R.D.) can be used to compare the density of a given material to that of water.  Water is the standard with a density of 1.0 g/mL, so all other densities are compared to water.
EQUATION: R.D. = Density of substance / Density of water

\subsubsection*{Materials}
Small empty kerosene lamp (kibatari) or other small bottle, rubber stopper/dry grass/maize cob*, plastic bag, water, honey, kerosene, cooking oil, straw/bamboo stick

\subsubsection*{Preparation Procedure}
\begin{enumerate}
\item{Find a person who can make a small kerosene burner (kibatari).} 
\item{Stretch the plastic bag over the opening of the kibatari and pass a straw/bamboo stick through to act as a rubber stopper.} 
\end{enumerate}

\subsubsection*{Activity Procedure}
\begin{enumerate}
\item{Weigh the small kibatari with its stopper and plastic bag in air (variable \ce{M0}).} 
\item{Fill the small kibatari with water and weigh it (variable \ce{M1}).} 
\item{Pour out the water from the small kibatari and dry it with a piece of clean cloth or tissue.} 
\item{Fill the small kibatari with another liquid, like honey, kerosene, or cooking oil, and weigh it again. (Variable \ce{M2})}
\item{Repeat steps 3 and 4 for other liquids}
\end{enumerate}

\subsubsection*{Results and Conclusion}
Mass of density bottle = \ce{M0}\\
Mass of density bottle with water = \ce{M1}\\
Mass of the density bottle with liquid A = \ce{M2}\\
Relative density of liquid A = Mass of liquid A/Mass of water or\\
$\frac{M_2-M_0}{M_1-M_0}$
We use this formula to find the relative density of any liquid.

\subsubsection*{Clean Up Procedure}
\begin{enumerate}
\item{Collect all the used materials, cleaning and storing items that will be used later.} 
\end{enumerate}

\subsubsection*{Discussion Questions}
\begin{enumerate}
\item{Determine the mass of the water.} 
\item{Calculate the volume of the density bottle and the relative density of honey.} 
\end{enumerate}

\subsubsection*{Notes}
Instead of small kibatari, a tin can or any small bottle can be used.  A density bottle can also be used to find the density of a solid, liquid and granules.
When finding relative density, units must be considered carefully.  Before you find the ratio of liquid density to water density, be sure that the units are the same.  g/mL or kg/L are the common units.

\subsection{Construction and Use of a Hydrometer}

\subsubsection*{Learning Objectives}
\begin{itemize}
\item{To construct a simple hydrometer.} 
\item{To explain the use of a hydrometer.} 
\item{To calibrate a hydrometer.} 
\item{To use a hydrometer to measure the density of various liquids.} 
\end{itemize}

\subsubsection*{Background Information}
Each liquid has a different density. The level at which an object will float in a liquid depends on the density of that liquid, so the different densities of liquids can be observed and measured by observing the level at which an object floats in them.  

\subsubsection*{Materials}
Plastic straw, small piece of plastic bag, dry sand, several containers to hold liquids, marker pen, rubber band or thread, ruler, water, kerosene, honey, any other liquids

\subsubsection*{Preparation Procedure}
\begin{enumerate}
\item{Close one end of the straw with the plastic bag and secure it with the rubber band or thread so that water cannot enter.} 
\item{Place the straw in water with the plastic bag side down.} 
\item{Pour sand into the straw until the bottom is heavy enough that the straw floats upright.} 
\end{enumerate}

\subsubsection*{Activity Procedure}
\begin{enumerate}
\item{Place the straw in water so that it floats upright without touching the bottom or leaning.} 
\item{Use the marker pen to mark the water level on the outside of the straw. Label this mark as 1.0.} 
\item{Place the straw in a container of kerosene so that it floats upright without touching the bottom or leaning to one side.} 
\item{Use the marker to mark the kerosene level on the outside of the straw. Label this mark as 0.8.}
\item{Remove the straw from the kerosene and clean it. Be careful not to get any liquid inside the straw.} 
\item{Use a ruler to draw an accurate scale on the straw, using the 1.0 and 0.8 marks as starting points. Begin by making a mark directly between them as 0.9, etc.} 
\item{When the scale is complete (both above 1.0 and below 0.8), use the designed hydrometer to measure the densities of other liquids by reading the mark at the level of the liquid.} 
\end{enumerate}

\subsubsection*{Results and Conclusion}
The straw, when properly weighted, will float upright in any liquid and will therefore provide a good surface to write levels of liquids. The density of water is known as 1.0 and is relatively constant. Kerosene is also known as 0.8 and will not vary from place to place. By writing both of these points on the hydrometer and the respective floating levels, we can create a scale extending up from 1.0 and down from 0.8. This can then be used to measure the densities of other liquids.  

\subsubsection*{Clean Up Procedure}
\begin{enumerate}
\item{Return all liquids to their proper containers.} 
\item{Clean the outside of the hydrometer.} 
\item{Return all materials to their proper places.} 
\end{enumerate}

\subsubsection*{Discussion Questions}
\begin{enumerate}
\item{What liquids have a high density?}
\item{What liquids have a low density?}
\item{Why, when reading a hydrometer, do the small numbers appear at the top of the scale?}
\end{enumerate}

\subsubsection*{Notes}
Be careful not to get any liquid in the straw. If the sand becomes wet, the hydrometer will not work again until it dries.  

\subsection{Applications of Material Densities}

\subsubsection*{Learning Objectives}
\begin{itemize}
\item{To observe the difference between densities of different liquids and solids.} 
\item{To design a density tower using a variety of liquids with different densities.} 
\end{itemize}

\subsubsection*{Background Information}
Materials can usually be distinguished from each other by their densities.  Normally, less dense materials float on denser liquids.  When liquids are placed in a container, the heavier (denser) liquids sink while the lighter (less dense) liquids float.  A density tower can be designed using liquids of different densities, even if they are soluble.

\subsubsection*{Materials}
Water, honey, glycerine, different food color, cooking oil, spirit, kerosene, beakers*, test tubes*, syringes, small pieces of wood, small pieces of rubber, and small pieces of metal

\subsubsection*{Preparation Procedure}
\begin{enumerate}
\item{Find a test tube or syringe.}
\item{Place each liquid into a beaker so it can easily be obtained by students.} 
\end{enumerate}

\subsubsection*{Activity Procedure}
\begin{enumerate}
\item{Place a small amount of honey into the test tube.} 
\item{Slowly add glycerine to the test tube.} 
\item{Slowly pour water into the test tube.} 
\item{Slowly add methylated spirit to the test tube.} 
\item{Add cooking oil into the test tube.} 
\item{Slowly add kerosene to the test tube.} 
\item{Put a small piece of wood, rubber and iron into the test tube.}
\item{Observe the positions of the liquids and solids relative to each other.}
\end{enumerate}

\subsubsection*{Results and Conclusion}
The liquids with a higher density will sink, while the liquids with less density will float. When the solid materials are dropped into the test tube, the solid material will rest in the liquid that has a relatively equal density.  

\subsubsection*{Clean Up Procedure}
\begin{enumerate}
\item{Collect all materials and return them to their proper place.} 
\end{enumerate}

\subsubsection*{Discussion Questions}
\begin{enumerate}
\item{Why should liquid be poured slowly into the test tube?}
\item{What happens when the wood, iron, and rubber are placed into test tubes?} 
\end{enumerate}

\subsubsection*{Notes}
Heavier density liquid should be poured into the test tube first, followed by relatively less dense liquids. Pouring the liquids should be done slowly to avoid mixing of liquid. Food colour can also be added to colorless liquid such as water, kerosene, and glycerine.  

\section{Archimedes' Principle}

\subsection{Construction of Eureka Can}

\subsubsection*{Learning Objectives}
\begin{itemize}
\item{To construct a eureka can and use it to measure volumes of irregular objects.} 
\end{itemize}

\subsubsection*{Materials}
Plastic bottles of 500 mL or 1000 mL, straws* or a syringe needle cap, a knife/razor blade, heat source*, cellotape or super glue, nail, and a wire

\subsubsection*{Preparation Procedure}
\begin{enumerate}
\item{Cut the plastic bottle in half and use the bottom part.} 
\item{Heat a sharp pointed end of the nail.} 
\item{Using the heated sharp point of the nail, make a small hole about 5 cm from the top of the cut plastic bottle.} 
\item{Cut a piece of straw about 5 cm long or use the syringe needle cap.} 
\item{Insert the piece of straw into the hole. Make sure the piece of straw is tightly fixed with cellotape or super glue.} 
\end{enumerate}

\subsubsection*{Activity Procedure}
\begin{enumerate}
\item{Use the constructed eureka can to overflow different liquids and to measure the volume of liquid displaced when an object floats in it.} 
\end{enumerate}

\subsubsection*{Clean Up Procedure}
\begin{enumerate}
\item{Collect all the used materials, cleaning and storing items that will be used later.} 
\end{enumerate}

\subsubsection*{Discussion Questions}
\begin{enumerate}
\item{What is the relationship between the weight of a floating object and the weight of water it displaces?}
\item{What is the reason for using a Eureka Can instead of a normal bottle?}
\end{enumerate}

\subsubsection*{Notes}
A Eureka Can is typically used to move a certain volume from the can into another container.  When an object floats in a liquid, it displaces its own weight in water.  If the level of water in the Eureka can is at the level of the spout, the water displaced will flow through the spout into another container.  This water can then be measured on a beam balance to find the weight of the object.
