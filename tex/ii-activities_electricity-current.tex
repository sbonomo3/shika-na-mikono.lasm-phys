\section{Current Electricity}

\subsection{Conductors and Insulators}

\subsubsection*{Learning Objectives}
\begin{itemize}
\item{To distinguish between conductors and insulators} 
\end{itemize}

\subsubsection*{Materials}
A nail, small piece of a bucket lid, aluminium foil from an empty cigarette packet, cotton thread, connecting wires, two dry cell, light bulb


\subsubsection*{Activity Procedure}
\begin{enumerate}
\item{Connect a bulb, dry cell, and a nail in a series using the connecting wire. Observe the effects.} 
\item{Replace the nail with aluminium foil, a cotton thread, and a small piece of a bucket cap, one after the other. Observe the effects.} 
\end{enumerate}

\subsubsection*{Results and Conclusion}
The bulb lights when it is connected with a nail and aluminium foil, indicating that there is a flow of current. The bulb does not light when it is connected with a cotton thread and a small piece of a bucket lid, indicating that a current does not flow. A nail and aluminium foil are conductors of electricity while a piece of a cotton thread and a piece of bucket cap are insulators -- they do not conduct electricity.

\subsubsection*{Clean Up Procedure}
\begin{enumerate}
\item{Collect all the used materials, cleaning and storing items that will be used later.} 
\end{enumerate}

\subsubsection*{Discussion Questions}
\begin{enumerate}
\item{What is a conductor?}
\item{Give four examples of (a) conductors (b) insulators.} 
\end{enumerate}

\subsubsection*{Notes}
Metals like copper, aluminium, iron are used for connecting electric current. When electrons flow through the metal they reach the bulb and so it lights up. Such metals are called Conductors. On the other hand, plastics, wood, cotton threads are not used because electrons are not free flow through them so the bulb will not light. Such materials are called insulators. 


\subsection{Finding Electric Circuit Components}

\subsubsection*{Learning Objectives}
\begin{itemize}
\item{To identify various components used in electric circuits} 
\item{To find circuit components locally} 
\end{itemize}

\subsubsection*{Background Information}
All of the electrical components studied in secondary school are common devices. These include resistors, capacitors, wires, motors, rheostats, switches, diodes, transistors, transformers, speakers, inductors, bulbs, etc. These components are common enough that they can be found in varying numbers and combinations in any electrical device. 

\subsubsection*{Materials}
broken radio, broken car stereo, broken computer, broken phone charger, broken disc drive, etc., pliers, screw driver, soldering iron (\textit{optional})

\subsubsection*{Hazards and Safety}
\begin{itemize}
\item{If you are using a soldering iron, be careful not to touch it or touch someone else with it. It can very quickly cause second degree burns.} 
\item{After breaking and removing components, dispose of any sharp pieces.} 
\item{NEVER open a component which is connected to an electrical source!} 
\end{itemize}

\subsubsection*{Preparation Procedure}
\begin{enumerate}
\item{Ask people to bring any broken electrical devices.} 
\item{Go to a radio repair shop in town to find any broken or old components.} 
\end{enumerate}

\subsubsection*{Activity Procedure}
\begin{enumerate}
\item{Open any electrical device.} 
\item{Identify all of the components visible inside the device.} 
\item{Remove as many components as possible from the devices and place them in a container. Pliers can be used to retrieve most things, but sometimes a soldering iron will be necessary to melt the flux holding some components into their boards.} 
\end{enumerate}

\subsubsection*{Results and Conclusion}
Circuit components can be found almost anywhere and can be used to perform many activities.

\subsubsection*{Clean Up Procedure}
\begin{enumerate}
\item{Collect all useful components in a container and store for later use.} 
\item{Dispose of any unused pieces.} 
\end{enumerate}

\subsubsection*{Discussion Questions}
\begin{enumerate}
\item{What components were you able to retrieve, and from what devices?}
\item{What components were you not able to retrieve, and where might you find them?}
\end{enumerate}

\subsubsection*{Notes}
This should be an ongoing activity for any school. Circuit components are always needed as they are easily destroyed in the laboratory. Keep looking for more things to take apart. 

\subsection{Measuring Emf of a Cell}

\subsubsection*{Learning Objectives}
\begin{itemize}
\item{To identify the terminals of a cell} 
\item{To use a voltmeter or multimeter in parallel} 
\item{To measure the electromotive force of a single cell or battery} 
\end{itemize}

\subsubsection*{Background Information}
Electromotic force (Emf) -- also called voltage or potential difference -- is the force from a cell or battery to move electric current through a circuit. A voltmeter reads the difference in voltage between two points in a circuit. If the difference in voltage is measured across a cell, the amount of voltage used in the entire circuit, from the positive terminal of the cell to the negative terminal, is being measured. This is the same as measuring the amount of voltage supplied by a cell, or the Emf. 

\subsubsection*{Materials}
Working dry cells, speaker wire / connecting wire, multimeter or voltmeter



\subsubsection*{Activity Procedure}
\begin{enumerate}
\item{Set the multimeter to the DCV (direct current voltage) setting.} 
\item{Connect the terminals of the multimeter to the terminals of the battery so that the multimeter displays a voltage level.} 
\item{Adjust the voltage magnitude on the multimeter until the voltage displayed is a clear, readable value.} 
\item{Connect two batteries in series and measure the total Emf.}
\item{Connect two batteries in parallel and measure the total Emf.}
\end{enumerate}

\subsubsection*{Results and Conclusion}
If a new battery is being used, the voltmeter should display the full Emf of a cell, which is 1. 5~V. If an older battery is being used, the Emf may show a slightly lower value. Students will see that the voltage (Emf) measured across a single cell shows all of the force of the cell without losing any through resistors. Also, the Emf of batteries in series is more than that of batteries in parallel.

\subsubsection*{Clean Up Procedure}
\begin{enumerate}
\item{Turn off the multimeter.} 
\item{Collect all the used materials, storing items that will be used later.} 
\end{enumerate}

\subsubsection*{Discussion Questions}
\begin{enumerate}
\item{What causes the force of the cell?}
\item{What is moving through the circuit?}
\item{Why is a voltmeter connected in parallel across a cell in order to measure Emf?}
\item{What is the difference between the Emf of batteries in series and parallel?}
\item{How would you calculate the Emf of batteries in series? In parallel?}
\end{enumerate}

\subsubsection*{Notes}
Electromotive force (Emf) is a kind of voltage. The other kind of voltage is Potential Difference and is used to measure the voltage difference across a resistor. Emf is always the same for a battery; it does not depend on the circuit it is connected to. Connecting batteries in series simply adds the forces of all the batteries together. Connecting batteries in parallel provides the same voltage as the battery with the largest voltage.
