\section{Light}

\subsection{Light in a Straight Line}

\subsubsection*{Learning Objectives}
\begin{itemize}
\item{To explain the concept of light rays and beam of light} 
\item{To demonstrate that light travels in straight line} 
\end{itemize}

\subsubsection*{Materials}
Torch, kerosene lamp, 3 cardboard rectangles, 6 large books, light source, iron nail

\subsubsection*{Preparation Procedure}
\begin{enumerate}
\item{Cut 3 rectangular pieces of cardboard. Make a hole at the center of each piece of cardboard using a nail.} 
\end{enumerate}

\begin{figure}
\begin{center}
\def\svgwidth{350pt}
\input{./img/prop-of-light.pdf_tex}
\caption{Experiment to show that light travels in straight lines}
\label{fig:prop-of-light}
\end{center}
\end{figure}

\subsubsection*{Activity Procedure}
\begin{enumerate}
\item{Arrange the cardboard pieces in between two book so they stand upright. The cardboard should be at a distance of at least 45 cm apart in straight so that the holes are aligned.} 
\item{Place a light source 30 cm from the first piece of cardboard. Have an observer stand at the other end of the table.} 
\item{Look through the series of holes and see the light source.}
\item{Slightly displace one of the pieces of cardboard and again look through the holes.}
\end{enumerate}

\subsubsection*{Clean Up Procedure}
\begin{enumerate}
\item{Collect all the used materials, cleaning and storing items that will be used later.} 
\end{enumerate}

\subsubsection*{Discussion Questions}
\begin{enumerate}
\item{Why should the holes be aligned?}
\end{enumerate}

\subsubsection*{Notes}
This activity is best done at night or in a dark room so that the light can be seen clearly through the holes.


\subsection{Pin Hole Camera}

\subsubsection*{Learning Objectives}
\begin{itemize}
\item{To investigate the law of light} 
\item{To observe that light travels in straight line} 
\end{itemize}

\subsubsection*{Background Information}
Light rays travel in a straight line.  When the rays of light from a source pass through a small hole, the image of the source (any object producing or reflecting light) can be seen, inverted, on the other side of the hole.  A simple, closed box can be used to demonstrate this property of light.  This instrument is called a pinhole camera and it demonstrates the basis of photography.

\subsubsection*{Materials}
Cardboard box, plain paper, candle, kerosene, pin, glue, matches

\subsubsection*{Hazards and Safety}
\begin{itemize}
\item{Kerosene is highly flammable and should be handled with care.} 
\end{itemize}

\subsubsection*{Preparation Procedure}
\begin{enumerate}
\item{Cut one side of an empty box and on the opposite side make a small hole using a pin.} 
\item{Soak a piece of plain paper in kerosene to make it transparent.} 
\item{Using glue, cover the open side of the box with the plain paper.} 
\item{Light the candle.} 
\end{enumerate}

\begin{figure}
\begin{center}
\def\svgwidth{250pt}
\input{./img/pinhole-camera.pdf_tex}
\caption{Construction of a pinhole camera}
\label{fig:pinhole-camera}
\end{center}
\end{figure}

\subsubsection*{Activity Procedure}
\begin{enumerate}
\item{Place the candle in front of the box on the side with the small hole.} 
\item{Observe and record the image of the candle on the plain paper.} 
\end{enumerate}

\subsubsection*{Results and Conclusion}
Rays of light travel in a straight line. Tee observed image is inverted as the result of the path of light rays from the object to the paper -- the rays cross at the pin hole.

\subsubsection*{Clean Up Procedure}
\begin{enumerate}
\item{Collect all the used materials, cleaning and storing items that will be used later. No special waste disposal is required.} 
\end{enumerate}

\subsubsection*{Discussion Questions}
\begin{enumerate}
\item{What properties of light allow the image to appear?}
\item{Why is the image of the candle inverted?}
\end{enumerate}

\subsubsection*{Notes}
The hole must be very small for the pin hole camera to work. A large hole will create a blurred image.  


\subsection{Laws of Reflection}

\subsubsection*{Learning Objectives}
\begin{itemize}
\item{To verify the laws of reflection in a plane mirror}
\end{itemize}

\subsubsection*{Materials}
plane mirror, pins, thick cardboard, protractors and rulers, white paper, pen, pencil

\subsubsection{Activity Procedure}
\begin{enumerate}
\item{Place a plane mirror vertically on a sheet of white paper on top of the cardboard.}
\item{Draw a line along the back of the mirror.}
\item{Construct a perpendicular line to the line on which the mirror stands.}  
\item{Draw a line making an angle of incidence ί from the normal.}
\item{Insert two pins on the line drawn which makes an angle ί with the normal.}
\item{Look into the mirror such that the images of the pins look as if they are in straight line.}
\item{Insert two more pins so that they are in line with the images of the first two pins.  These two more pins mark the path of the reflected ray.}
\item{Remove the pins and draw lines joining the marks of the pins.}
\item{Using a protractor measure and record the angle between the reflected ray and the normal.}
\end{enumerate}

\subsubsection*{Results and Conclusion}
This practical is used to verify the Laws of Reflection and to observe and describe images formed in a plane mirror.  It will be seen that the angles of reflection are equal to the angles of incidence in each case.
